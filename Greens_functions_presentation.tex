% rubber: set program xelatex

% The theme used for this presentation is matze's mtheme, which can be
% found at https://github.com/matze/mtheme

\newif\ifhandout

\handouttrue

\ifhandout
    \documentclass[12 pt, compress, handout, intlimits]{beamer}

    \setbeamertemplate{note page}[plain]
    \setbeameroption{show notes}% on second screen=bottom}
\else

    \documentclass[12 pt, compress, intlimits]{beamer}

\fi

\usetheme{m}

\usepackage{mymacros}
\usepackage[retainorgcmds]{IEEEtrantools}
\setlength{\IEEEnormaljot}{9pt}

\renewcommand{\d}[1]{\ensuremath{\operatorname{d}}\!{#1}}

\usepackage{graphicx}
\usepackage{booktabs}
\usepackage[scale=2]{ccicons}
\usepackage{array}

%\usepgfplotslibrary{dateplot}

\renewcommand{\vect}[1]{\vec{#1}}
\renewcommand{\L}{\mathcal{L}}

\setbeamercovered{transparent}

\title{Green's functions}
\subtitle{A short introduction}
\date{\today}
\author{Chris Deimert}
\institute{Department of Electrical and Computer Engineering, University of Calgary}


\begin{document}

\maketitle

\note{}

\begin{frame}[fragile]
    \frametitle{Outline}
    \tableofcontents
\end{frame}

\note{}

%\begin{frame}[fragile]
%    \frametitle{Vectors and pseudo-vectors}
%
%    \begin{center}
%        $ \vect{E} $ is a \alert{vector} under reflection
%    \end{center}
%    \begin{figure}
%        \centering
%        \includegraphics[width=0.8\textwidth]{Reflection_e_field.pdf}
%    \end{figure}
%    \note{
%        If we mirror space, $ \vect{E} $ mirrors as expected. 
%        This is because $ \vect{E} $ is a true vector.
%    }
%\end{frame}

\section{Basic idea}
\label{sec:basic_idea}

\note{}

\begin{frame}[fragile]
    \frametitle{What is a Green's function?}
    
    Linear equation to solve:
    \begin{align*}
        \L u(x) &= f(x)
    \end{align*}
    
    \pause

    Green's function is impulse response:
    \begin{align*}
        \L G(x,x') &= \delta(x - x')
    \end{align*}

\end{frame}

\note{
\begin{itemize}
\item
    Most EM problems are described by linear (differential) equations with some source function $ f(x) $.
\item
    The Green's function is the solution when $ f(x) $ is an impulse located at $ x' $.
\item
    Generalization of impulse response from signal processing.
\end{itemize}
}

\begin{frame}[fragile]
    \frametitle{Why is it useful?}

    Can find the solution directly for any $ f(x) $:
    \begin{align*}
        u(x) &= \int G(x,x') f(x') \d x
    \end{align*}
    because
    \begin{align*}
        \L u(x) &= \int \L G(x,x') f(x') \d x = \int \delta(x - x') f(x) \d x = f(x)
    \end{align*}
    

    
\end{frame}

\note{
\begin{itemize}
\item
    Once we know the Green's function for a problem, we can find the solution for any source.
\item
    Kind of like splitting the source $ f(x) $ into little impulses and then adding up (integrating) the response to each impulse.
\item
    Generalization of convolution from signal processing.
\item
    Can be quicker than adding up infinite series of orthogonal functions.
\end{itemize}
}

\begin{frame}[fragile]
    \frametitle{Summary}
    \begin{center}
    Some important thing.
    \end{center}
\end{frame}


\begin{frame}[fragile]
    \frametitle{Introductory resources}
    Balanis (2012), \emph{Advanced engineering electromagnetics}. 
    Less rigorous, but good for getting the key ideas. Good place to start.

    Folland (1992), \emph{Fourier analysis and its applications}. 
    Fairly rigorous.
    Chapter on generalized functions is particularly nice.

    Dudley (1994), \emph{Mathematical foundations for electromagnetic theory}.
    Fairly rigorous. 
    Great introduction to 1D Green's functions: deals with subtleties other books leave out.
    
\end{frame}    

\note{}

\begin{frame}[fragile]
    \frametitle{Advanced resources}
    Collin (1990), \emph{Field theory of guided waves}. 
    Huge chapter on Green's functions. Emphasis on dyadics.

    Morse and Feshback, \emph{Methods of theoretical physics}.
    Another big, detailed reference. Emphasis on theory and insights.

    Warnick (1996), ``Electromagnetic Green functions using differential forms.''
    For the differential forms inclined.

\end{frame}    

\note{}

\section{Solution methods}
\label{sec:solution_methods}

\note{}

\begin{frame}[fragile]
    \frametitle{Solution methods}
    Boundary condition approaches:
    \begin{enumerate}
    \item
        Green's function gives particular solution; add homogeneous solution to find boundary conditions.
        Easier to set up, but requires extra work to deal with BC's.
    \item
        Green's function includes BC's.
        Harder to set up, but gives full solution including BC's.
    \end{enumerate}
\end{frame}

\note{}

\begin{frame}[fragile]
    \frametitle{Solution methods}
     Solving Green's function approaches:
    \begin{enumerate}
    \item
        Direct solution. (Great if it's possible.)
    \item
        Eigenvalue expansion. (Works every time.)
    \end{enumerate}
\end{frame}

\note{}

\section{Applications}
\label{sec:applications}

\note{}

\begin{frame}[fragile]
    \frametitle{Applications}
    \begin{itemize}
    \item
        Born approximation for scattering?
    \item
        Perturbation theory?
    \item
        Propagator/Huygen's principle?
    \end{itemize}
\end{frame}

\note{}

\end{document}
