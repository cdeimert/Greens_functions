% rubber: set program xelatex

% The theme used for this presentation is matze's mtheme, which can be
% found at https://github.com/matze/mtheme

\newif\ifhandout

\handoutfalse
\handouttrue

\ifhandout
    \documentclass[12 pt, compress, handout, intlimits]{beamer}

    \setbeamertemplate{note page}[plain]

    \setbeameroption{show notes}% on second screen=bottom}
\else

    \documentclass[12 pt, compress, intlimits]{beamer}

\fi

\usetheme{m}

\usepackage{mymacros}
\usepackage[retainorgcmds]{IEEEtrantools}
\setlength{\IEEEnormaljot}{9pt}

\renewcommand{\d}{\ensuremath{\operatorname{d}}}

\usepackage{graphicx}
\usepackage{booktabs}
\usepackage[scale=2]{ccicons}
\usepackage{array}

\usepackage{xcolor}
\usepackage{mathtools}
\usepackage{empheq}

\newcommand{\highlight}[1]{\colorbox{mLightBrown!75}{$\displaystyle{#1}$}}
\newcommand{\mygreenbox}[1]{\colorbox{mLightBrown!75}{\hspace{1em}#1\hspace{1em}}}

%\usepackage{sourcesanspro}

%\usepgfplotslibrary{dateplot}

%\usefonttheme[onlymath]{serif}
%\renewcommand{\vect}[1]{\vec{#1}}
\renewcommand{\L}{\mathcal{L}}

\setbeamercovered{transparent}

\title{Green's functions}
\subtitle{A short introduction}
\date{\today}
\author{Chris Deimert}
\institute{Department of Electrical and Computer Engineering, University of Calgary}


\begin{document}

\maketitle

\note{
    \begin{itemize}
    \item
        This is intended as a quick overview of Green's functions for electrical engineers.
    \item
        Green's functions are a huge subject: it's easy to get overwhelmed by calculation techniques. 
    \item
        Focus here will be on intuition/understanding and awareness of some key techniques.
    \item
        Lots of further reading provided at the end.
    \end{itemize}
}

\begin{frame}[fragile]
    \frametitle{Outline}
    \tableofcontents
\end{frame}

\note{}

\section{Introduction}
\label{sec:introduction}

\note{
\begin{itemize}
\item
    Fortunately, the basic idea of Green's functions is really simple.
\item
    You've actually used them before!
\end{itemize}
}

\begin{frame}[fragile]
    \frametitle{What is a Green's function?}
    
    Linear equation to solve:
    \begin{align*}
        \L u(x) &= f(x)
    \end{align*}
    
    \pause

    Green's function is the \textbf{impulse response}:
    \begin{align*}
        \L G(x,x') &= \delta(x - x')
    \end{align*}

\end{frame}

\note{
\begin{itemize}
\item
    Most EM problems are described by linear (differential) equations with some source/driving function $ f(x) $.
\item
    The Green's function is the solution when the source $ f(x) $ is an impulse located at $ x' $.
\item
    Can think of it as a generalization of the impulse response from signal processing.
\end{itemize}
}

\begin{frame}[fragile]
    \frametitle{Why is it useful?}
    
    \begin{align*}
        \delta(x - x') &\xrightarrow{\quad \L^{-1} \quad} G(x,x')
    \end{align*}
 
    \pause

    \begin{align*}
        f(x) = \int \delta(x - x') f(x') \d x \xrightarrow{\quad \L^{-1} \quad} \int G(x,x') f(x') \d x
    \end{align*}
  
    %Can find the solution directly for any $ f(x) $:
    %\begin{align*}
    %    u(x) &= \int G(x,x') f(x') \d x
    %\end{align*}
    %\begin{align*}
    %    \L u(x) &= \int \L G(x,x') f(x') \d x = \int \delta(x - x') f(x) \d x = f(x)
    %\end{align*}
    
\end{frame}

\note{
\begin{itemize}
\item
    Once we know the Green's function for a problem, we can find the solution for any source $ f(x) $.
\item
    Impulses $ \delta(x - x') $ produce a response $ G(x,x') $.
\item
    We can split the source $ f(x) $ up into a sum (integral) of impulses $ \delta(x - x') $.
\item
    Then the response to $ f(x) $ is just a weighted sum (integral) of impulse responses.
\end{itemize}
}

\begin{frame}[fragile]
    \frametitle{Why is it useful?}

    \begin{align*}
        \L u(x) &= f(x)
    \end{align*}
    \begin{align*}
        \L G(x,x') &= \delta(x - x')
    \end{align*}
    \begin{empheq}[box=\mygreenbox]{align*}
        u(x) = \int G(x,x') f(x') \d x
    \end{empheq}

\end{frame}

\note{
\begin{itemize}
\item
    Once we know the Green's function, we have an explicit formula for the solution $ u(x) $ for any source function $ f(x) $.
\end{itemize}
}

\begin{frame}[fragile]
    \frametitle{Familiar Green's functions}

    Impulse response of a LTI system:
    \begin{align*}
        y(t) &= \int_{-\infty}^{\infty} x(t') \alert{h(t - t')} \d t'
    \end{align*}

    \pause
    E.g., for an RL-circuit:
    \begin{align*}
        G(t,t') &= h(t - t') = u(t - t') e^{-\alpha (t - t')}
    \end{align*}
    
\end{frame}

\note{
\begin{itemize}
\item
    In electrical engineering, we've seen Green's functions before.
\item
    Impulse response $ h(t - t') $ from linear system theory is an example of a Green's function.
    \begin{align}
        G(t,t') & = h(t - t')
    \end{align}
\item
    Usually find $ h(t - t') $ using Fourier transform of the transfer function.
\end{itemize}
}

\begin{frame}[fragile]
    \frametitle{Familiar Green's functions}

    Poisson's equation:
    \begin{align*}
        \nabla^2 V(\vect{r}) &= - \frac{\rho(\vect{r})}{\epsilon_0}
    \end{align*}
    \begin{align*}
        V(\vect{r}) &= \iiint \alert{\frac{1}{4 \pi \epsilon_0 \left| \vect{r} - \vect{r}' \right|^2}} \rho(\vect{r}') \d^3 \vect{r}'
    \end{align*}
\end{frame}

\note{
    \begin{itemize}
    \item
        Green's function for Poisson's equation is
        \begin{align*}
            G(\vect{r}, \vect{r}') &= \frac{1}{4 \pi \epsilon_0 \left| \vect{r} - \vect{r}' \right|^2}
        \end{align*}
    \end{itemize}
}

\begin{frame}[fragile]
    \frametitle{Familiar Green's functions}
    
    Helmholtz equation:
    \begin{align*}
        \left( \nabla^2 + k^2 \right) A_z(\vect{r}) &= -J_z(\vect{r})
    \end{align*}
    \begin{align*}
        A_z(\vect{r}) &= \iiint \alert{\frac{e^{-jk \left| \vect{r} - \vect{r}' \right|}}{4 \pi \left| \vect{r} - \vect{r}' \right|}} J_z\left( \vect{r}' \right) \d^3 \vect{r}'
    \end{align*}
\end{frame}

\note{
    \begin{itemize}
    \item
        Green's function for the Helmholtz equation is
        \begin{align*}
            G(\vect{r}, \vect{r}') &= \frac{e^{-jk \left| \vect{r} - \vect{r}' \right|}}{4 \pi \left| \vect{r} - \vect{r}' \right|}
        \end{align*}
    \end{itemize}
}

\begin{frame}[fragile]
    \frametitle{Familiar Green's functions}
    
    Our goal:
    \begin{itemize}
    \item
        Derive these expressions.
    \item
        Generalize to other problems and boundary conditions.
    \end{itemize}
    
\end{frame}

\note{
}

\section{Generalized functions}
\label{sec:generalized_functions}

\note{
    \begin{itemize}
    \item
        Delta functions play a key role in Green's functions (and electrical engineering in general).
    \item
        Worth seeing how they are rigorously defined before moving on.
    \item
        See Folland (1992), \emph{Fourier analysis and its applications}, Chapter 9 for more.
    \end{itemize}
    
}

\begin{frame}[fragile]
    \frametitle{Poor definition}
    
    Common ``definition'':
    \begin{align*}
        \int_{-\infty}^{\infty} \delta(x-x_0) &= 1
    \end{align*}
    \begin{align*}
        \delta(x-x_0) &= \begin{cases} 0 & \text{for } x \neq x_0 \\ \infty & \text{for } x = x_0 \end{cases}
    \end{align*}
    
\end{frame}

\note{
\begin{itemize}
\item
    Often see definitions like ``$ \delta(x-x_0) $ is zero for $ x \neq x_0 $, but the area under it is 1.''
\item
    Might be okay intuitively, but very imprecise mathematically.
\end{itemize}
}

\begin{frame}[fragile]
    \frametitle{Better definition}

    \begin{itemize}
    \item
    $ \delta(x-x_0) $ is an operator, \textbf{not} a function!
    \pause
    \item
    Define it by the sifting property:
    \begin{align*}
        \delta_{x_0}[f] &= f(x_0)
    \end{align*}
    \pause\vspace*{-\baselineskip}
    \item
    \textbf{Symbolically}, write
    \begin{align*}
        \int_{-\infty}^{\infty} \delta(x - x_0) f(x) \d x &= f(x_0)
    \end{align*}
    \end{itemize}
    
\end{frame}

\note{
    \begin{itemize}
    \item
        Delta function is rigorously defined using Schwartz distribution theory.
    \item
        Basically, the delta function is not a function at all, it is a linear operator which takes a function and returns a number: the value of the function at $ x_0 $.
    \item
    I.e., \emph{the sifting property is the definition of the delta function.}
    \item
        Symbolically, we often write it as a function, but it's good to remember the proper definition in case anything fishy shows up.
    \item
        Remember, $ \delta(x - x') $ has no meaning as a stand-alone function.
        It only has meaning when it operates on a function.
    \end{itemize}
}

\begin{frame}[fragile]
    \frametitle{Delta function limits}
    
    \begin{align*}
        \lim_{n \to \infty} \phi_n(x) &= \delta(x)
    \end{align*}
    if and only if
    \begin{align*}
        \lim_{n \to \infty} \int_{-\infty}^{\infty} \phi_n(x) f(x) \d x &= f(0)
    \end{align*}
    
\end{frame}

\note{
\begin{itemize}
\item
    Often, we want to show that regular functions are equivalent to the delta function.
\item
    To do this in a reasonable way, we need to show that the sifting property is obeyed, usually in a limit.
\end{itemize}
}

\begin{frame}[fragile]
    \frametitle{Delta function limits}
    Example:
    \begin{align*}
        \frac{1}{2 \pi} \int_{-\infty}^{\infty} e^{j x t} \d t &= \delta(x)
    \end{align*}
    because
    \begin{align*}
        \frac{1}{2 \pi} \int_{-\infty}^{\infty} e^{-\epsilon^2 t^2} e^{j x t} \d t
    \end{align*}
    is a delta-function limit as $ \epsilon \to 0 $.

\end{frame}

\note{
    \begin{itemize}
    \item
        Example of a common, but perplexing expression for the Delta function.
    \item
        Can show that it's true by expressing it as a delta function limit.
        (Multiply by a Gaussian distribution and take the limit as standard deviation goes to $ \infty $.)
    \item
        Doing this proof is not a bad exercise if you're interested. Theorem 9.2 from Folland will make it manageable.
    \end{itemize}
    
}

\begin{frame}[fragile]
    \frametitle{Introductory resources}
    Balanis (2012), \emph{Advanced engineering electromagnetics}. 
    Less rigorous, but good for getting the key ideas.

    Folland (1992), \emph{Fourier analysis and its applications}. 
    Chapter on generalized functions is particularly nice.

    Dudley (1994), \emph{Mathematical foundations for electromagnetic theory}.
    Great introduction to 1D Green's functions: deals with subtleties that others ignore.

    Byron and Fuller (1992), \emph{Mathematics of classical and quantum physics}.
    Interesting alternative approach.
    
\end{frame}

\note{}

\begin{frame}[fragile]
    \frametitle{Advanced resources}
    Collin (1990), \emph{Field theory of guided waves}. 
    Huge chapter on Green's functions. Emphasis on dyadics.

    Morse and Feshback, \emph{Methods of theoretical physics}.
    Another big, detailed reference. Emphasis on theory and insights.

    Warnick (1996), ``Electromagnetic Green functions using differential forms.''
    For the differential forms inclined.

\end{frame}    

\note{}

\section{Solution methods}
\label{sec:solution_methods}

\note{}

\begin{frame}[fragile]
    \frametitle{Solution methods}
    Boundary condition approaches:
    \begin{enumerate}
    \item
        Green's function gives particular solution; add homogeneous solution to find boundary conditions.
        Easier to set up, but requires extra work to deal with BC's.
    \item
        Green's function includes BC's.
        Harder to set up, but gives full solution including BC's.
    \end{enumerate}
\end{frame}

\note{}

\begin{frame}[fragile]
    \frametitle{Solution methods}
     Solving Green's function approaches:
    \begin{enumerate}
    \item
        Direct solution. (Great if it's possible.)
    \item
        Eigenvalue expansion. (Works every time.)
    \end{enumerate}
\end{frame}

\note{}

\begin{frame}[fragile]
    \frametitle{Causality}
    \begin{itemize}
    \item
        Time-domain wave equation has a unique solution in the lossless case.
    \item
        Frequency-domain wave equation does not.
    \item
        Taking infinitesimally small loss is equivalent to assuming $ u(x) $ and $ u'(x) $ are zero at some initial time.
    \end{itemize}
\end{frame}

\note{}

\section{Applications}
\label{sec:applications}

\note{}

\begin{frame}[fragile]
    \frametitle{Applications}
    \begin{itemize}
    \item
        Born approximation for scattering?
    \item
        Perturbation theory?
    \item
        Propagator/Huygen's principle?
    \end{itemize}
\end{frame}

\note{}

\end{document}
